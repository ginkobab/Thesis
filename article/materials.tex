\section{Material and Methods}

\subsection{The environment}

\myparagraph{The brain column} 
The environment in which the agent acts is a scaled version of the Potjans and Diesmann column, where the number of neurons is 8000 instead of 80000. That was necessary to reduce computational time. The first only input to the agent was the reward, calculated as the negative squared root of the absolute difference between the firing rates of the excitatory neurons in the layers 2/3, 4, 5 and 6 recorded from the somatosensory cortex of a rabbit \cite{kock}, and the firing rates of the excitatory neurons in the same layers. Computing the reward in this way is not optimal, because the agent doesn't know in which layer the error was, but 
\myparagraph{The parameters} 

\myparagraph{The reward} 

\subsection{The agent}

\myparagraph{The algorithm} 

\myparagraph{Adaptation to the bandit problem} 


